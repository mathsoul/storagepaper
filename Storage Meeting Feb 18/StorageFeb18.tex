%%%%%%%%%%%%%%%%%%%%%%%%%%%%%%%%%%%%%%%%%
% Short Sectioned Assignment
% LaTeX Template
% Version 1.0 (5/5/12)
%
% This template has been downloaded from:
% http://www.LaTeXTemplates.com
%
% Original author:
% Frits Wenneker (http://www.howtotex.com)
%
% License:
% CC BY-NC-SA 3.0 (http://creativecommons.org/licenses/by-nc-sa/3.0/)
%
%%%%%%%%%%%%%%%%%%%%%%%%%%%%%%%%%%%%%%%%%

%----------------------------------------------------------------------------------------
%    PACKAGES AND OTHER DOCUMENT CONFIGURATIONS
%----------------------------------------------------------------------------------------




\documentclass[paper=a4, fontsize=15pt]{scrartcl} % A4 paper and 11pt font size

\usepackage{amssymb}
\usepackage[T1]{fontenc} % Use 8-bit encoding that has 256 glyphs
\usepackage{fourier} % Use the Adobe Utopia font for the document - comment this line to return to the LaTeX default
\usepackage[english]{babel} % English language/hyphenation
\usepackage{amsmath,amsfonts,amsthm} % Math packages
\usepackage{float}
\usepackage{lipsum} % Used for inserting dummy 'Lorem ipsum' text into the template
\usepackage{graphicx}
\usepackage{caption}
\usepackage{sectsty} % Allows customizing section commands
\allsectionsfont{\centering \normalfont\scshape} % Make all sections centered, the default font and small caps
\usepackage{listings}
\usepackage{color}
\usepackage{fancyhdr} % Custom headers and footers
\usepackage{multicol}
\usepackage{subcaption} %Insert multicolumns plots
\pagestyle{fancyplain} % Makes all pages in the document conform to the custom headers and footers
\usepackage{tikz} %the most complex and powerful tool to create graphic elements in LATEX
\usetikzlibrary{matrix,arrows,fit}
\tikzset{circarrow/.style={
        *->,
        shorten <=-2pt
    }
}
\fancyhead{} % No page header - if you want one, create it in the same way as the footers below
\fancyfoot[L]{} % Empty left footer
\fancyfoot[C]{} % Empty center footer
\fancyfoot[R]{\thepage} % Page numbering for right footer
\renewcommand{\headrulewidth}{0pt} % Remove header underlines
\renewcommand{\footrulewidth}{0pt} % Remove footer underlines
\setlength{\headheight}{13.6pt} % Customize the height of the header

\definecolor{mygreen}{rgb}{0,0.6,0}
\definecolor{mygray}{rgb}{0.5,0.5,0.5}
\definecolor{mymauve}{rgb}{0.58,0,0.82}

\lstset{ %
  backgroundcolor=\color{white},   % choose the background color; you must add \usepackage{color} or \usepackage{xcolor}
  basicstyle=\footnotesize,        % the size of the fonts that are used for the code
  breakatwhitespace=false,         % sets if automatic breaks should only happen at whitespace
  breaklines=true,                 % sets automatic line breaking
  captionpos=b,                    % sets the caption-position to bottom
  commentstyle=\color{mygreen},    % comment style
  frame=single,                    % adds a frame around the code
  keepspaces=true,                 % keeps spaces in text, useful for keeping indentation of code (possibly needs columns=flexible)
  keywordstyle=\color{blue},       % keyword style
  language=Matlab,                 % the language of the code
  numbers=left,                    % where to put the line-numbers; possible values are (none, left, right)
  numbersep=5pt,                   % how far the line-numbers are from the code
  numberstyle=\tiny\color{mygray}, % the style that is used for the line-numbers
  rulecolor=\color{black},         % if not set, the frame-color may be changed on line-breaks within not-black text (e.g. comments (green here))
 tabsize=2,                       % sets default tabsize to 2 spaces
  title=\lstname                   % show the filename of files included with \lstinputlisting; also try caption instead of title
}



\numberwithin{equation}{section} % Number equations within sections (i.e. 1.1, 1.2, 2.1, 2.2 instead of 1, 2, 3, 4)
\numberwithin{figure}{section} % Number figures within sections (i.e. 1.1, 1.2, 2.1, 2.2 instead of 1, 2, 3, 4)
\numberwithin{table}{section} % Number tables within sections (i.e. 1.1, 1.2, 2.1, 2.2 instead of 1, 2, 3, 4)

\setlength\parindent{0pt} % Removes all indentation from paragraphs - comment this line for an assignment with lots of text

%----------------------------------------------------------------------------------------
%    TITLE SECTION
%----------------------------------------------------------------------------------------



\newcommand{\horrule}[1]{\rule{\linewidth}{#1}} % Create horizontal rule command with 1 argument of height
\newcommand{\PP}{\mathcal{P}}

%\title{    
%\normalfont \normalsize 
%\textsc{UT Austin, Mccombs Business School} \\ [25pt] % Your university, school and/or department name(s)
%\horrule{0.5pt} \\[0.4cm] % Thin top horizontal rule
%\huge Homework 4 for Network Optimization\\ % The assignment title
%\horrule{2pt} \\[0.5cm] % Thick bottom horizontal rule
%}

%\author{Long Zhao lz4786} % Your name

%\date{\normalsize\today} % Today's date or a custom date

%\graphicspath{{/Users/zhaolong/Google Drive/Network Optimization/Homework1}}

\begin{document}

%\maketitle % Print the title

\section{HJB Equation}
Denote the price process as $P(t)$; the log seasonal factor as $S(t)$; the log deseasonal price process as $X(t)$; the storage process as $Q(t)$ . In other words, 
\begin{equation*}
  P(t) = \exp(X(t) + S(t)).
\end{equation*}
What's more, we know that $X(t)$ follows an OU process as following
\begin{equation}\label{Equation_dX(t)}
  dX(t) = \kappa(\alpha - X(t))dt + \sigma dW_t,
\end{equation}
where $W_t$ is a Brownian motion.


The value function, $V$, should be a function of the deseasonal price $x$, the seasonal factor $s$, and the storage level $q$. That is to say, $V(x,s,q)$ is the value function. In the holding region, $Q(t)$ remains. As a result, in the holding region, 
\begin{equation}\label{Equation_general_dV}
\begin{split}
  de^{-\beta t}V(X_t,S_t,Q_t) = &e^{-\beta t}\bigg(V_xdX_t + \frac{1}{2}V_{xx}(dX_t)^2 + V_sdS_t + \frac{1}{2}V_{ss}(dS_t)^2\\ 
  		+& V_{xs}dX_tdS_t - \beta V dt\bigg). 
\end{split}
\end{equation}

If $S(t)$ is deterministic and differentiable, we can assume it satisfies
\begin{equation*}
  dS(t) = f(S(t))dt.
\end{equation*}
Equation (\ref{Equation_general_dV}) turns out to be
\begin{equation*}
  de^{-\beta t} V(X_t,S_t,Q_t) = e^{-\beta t}\bigg(V_xdX_t + \frac{1}{2}V_{xx}(dX_t)^2 + f(S(t))V_sdt - \beta V\bigg).
\end{equation*}
Plug equation (\ref{Equation_dX(t)}) into previous equaiton, 
\begin{equation*}
  de^{-\beta t} V(X_t,S_t,Q_t) = e^{-\beta t}\left(\kappa(\alpha-X(t))V_x + \frac{1}{2}\sigma^2V_{xx} + f(S(t))V_s - \beta V\right)dt + \sigma e^{-\beta t}V_x dW_t.
\end{equation*}
Define operator $\mathcal{L}$ as
\begin{equation*}
  \mathcal{L} V(x,s,q) = \frac{1}{2} \sigma^2 V_{xx} + \kappa(\alpha - x) V_x - \beta V +f(s) V_s . 
\end{equation*}
As a conclusion, the HJB equation can be written as
\begin{equation*}
  \max \{\mathcal{L} V, -V_q + e^x - \mu(q), V_q -e^x - \lambda(q)\} = 0.
\end{equation*}

\newpage

\section{Solve $\mathcal{L}V = 0$.}
Assume $V(x,s,q) = g(x,q)h(s)$. When $s = 0$, there is no seasonal factor at all. In this case, we should have $h(0) = 1$. Plug it into $\mathcal{L} V = 0$ to have
\begin{equation*}
  h\left(\frac{1}{2} \sigma^2 g_{xx} + \kappa (\alpha - x) g_x - \beta g\right) + f(s) g h' = 0.
\end{equation*}
We can rearrange it as
\begin{equation*}
  \frac{\frac{1}{2} \sigma^2 g_{xx}(x,q) + \kappa (\alpha - x) g_x(x,q) - \beta g(x,q)}{g(x,q)} = \frac{-f(s) h'(s)}{h(s)}.
\end{equation*}
The left hand side is a function of $x$ and $q$ while the right hand side is a function of $s$. Therefore, the only possibility is that both are the same constant, denoted as $\eta$.
\begin{equation*}
\begin{split}
  &\frac{1}{2} \sigma^2 g_{xx}(x,q) + \kappa (\alpha - x) g_x(x,q) - (\beta + \eta) g(x,q) = 0\\
  &f(s) h'(s) + \eta h(s) = 0 \quad h(0) = 1
\end{split}
\end{equation*}
The solution to the second ODE is 
\begin{equation}\label{Equation_h}
  h(s) = \exp(-\eta \int_{0}^{s} \frac{1}{f(y)}dy)
\end{equation}

If we have $S(t) = \sin(t)$, then 
\begin{equation*}
	f(S(t)) = S'(t) = \cos(t) =  \sqrt{1-\sin(t)^2}   = \sqrt{1-S(t)^2}.
\end{equation*}
Namely, $f(s) = \sqrt{1-s^2}$. Substitute this back into equation (\ref{Equation_h}) to have
\begin{equation*}
  h(s) = \exp(-\eta \arcsin(s))
\end{equation*}

\textbf{One boundary condition is missing!} Assume we will hold on the line $x = x_0,~q = q_0$. Without another boundary condition, it is impossible to calculate $\eta$. It is not like on the line $x = x_0,~s = s_0$. Because it is impossible to buy (sell) when the facility is full (empty), the number of the boundary conditions are the same as the number of unknown variables.

\newpage
\section{A Question about Policy \& Value Iteration}

In the case without seasonality, we construct a Markov chain from the holding profit, 
\begin{equation*}
  \frac{1}{2}\sigma^2 V_{xx} + \kappa(\alpha - x) V_x - \beta V.
\end{equation*}

Firstly, we find an approximation of above expression. For example, in the case $x\geq \alpha$,
\begin{equation*}
\begin{split}
  &\frac{1}{2}\sigma^2 \frac{V(x+dx) - 2V(x) + V(x-dx)}{dx^2} + \kappa(\alpha - x) \frac{V(x) - V(x-dx)}{dx} - \beta V(x)\\
  = & \frac{1}{2}\frac{\sigma}{dx^2} V(x+dx) + \left(\frac{1}{2}\frac{\sigma^2}{dx^2} + \frac{\kappa(x-\alpha)}{dx}\right)V(x-dx) - \left(\frac{\sigma^2}{dx^2} + \frac{\kappa(x-\alpha)}{dx} + \beta\right)V(x)
\end{split}
\end{equation*}
Next, from this approximation, we set the transition probability of the Markov chain from $x$ to $x+dx$ as 
\begin{equation*}
  p_+ = \frac{\frac{1}{2}\frac{\sigma}{dx^2}}{\frac{\sigma^2}{dx^2} + \frac{\kappa(x-\alpha)}{dx} + \beta},
\end{equation*}
and from $x$ to $x-dx$ as
\begin{equation*}
  p_- = \frac{\frac{1}{2}\frac{\sigma^2}{dx^2} + \frac{\kappa(x-\alpha)}{dx}}{\frac{\sigma^2}{dx^2} + \frac{\kappa(x-\alpha)}{dx} + \beta}.
\end{equation*}

In this case, the sum of $p_+$ and $p_-$ is
\begin{equation*}
  p_+ + p_- =  \frac{\frac{\sigma^2}{dx^2} + \frac{\kappa(x-\alpha)}{dx}}{\frac{\sigma^2}{dx^2} + \frac{\kappa(x-\alpha)}{dx} + \beta},
\end{equation*}


\textbf{which is not 1.} Is it OK? How to understand it?

\section{A Question about the Storage Contract}
\begin{itemize}
  \item Start Date: 19th December 2012
  \item End Date: 18th December 2013
  \item Initial Inventory Level: 0
  \item Final Inventory Level: 0
  \item Capacity: 29.3 GWh
  \item Max Injection/Withdrawal: 1.465 GWh per day
  \item Underlying Gas Price: NBP (National Balancing Point) pence/therm
\end{itemize}



Is it a contract that enables one to use the storage facility?





\end{document}

