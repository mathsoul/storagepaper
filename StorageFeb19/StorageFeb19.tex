%%%%%%%%%%%%%%%%%%%%%%%%%%%%%%%%%%%%%%%%%
% Short Sectioned Assignment
% LaTeX Template
% Version 1.0 (5/5/12)
%
% This template has been downloaded from:
% http://www.LaTeXTemplates.com
%
% Original author:
% Frits Wenneker (http://www.howtotex.com)
%
% License:
% CC BY-NC-SA 3.0 (http://creativecommons.org/licenses/by-nc-sa/3.0/)
%
%%%%%%%%%%%%%%%%%%%%%%%%%%%%%%%%%%%%%%%%%

%----------------------------------------------------------------------------------------
%    PACKAGES AND OTHER DOCUMENT CONFIGURATIONS
%----------------------------------------------------------------------------------------




\documentclass[paper=a4, fontsize=15pt]{scrartcl} % A4 paper and 11pt font size

\usepackage{amssymb}
\usepackage[T1]{fontenc} % Use 8-bit encoding that has 256 glyphs
\usepackage{fourier} % Use the Adobe Utopia font for the document - comment this line to return to the LaTeX default
\usepackage[english]{babel} % English language/hyphenation
\usepackage{amsmath,amsfonts,amsthm} % Math packages
\usepackage{float}
\usepackage{lipsum} % Used for inserting dummy 'Lorem ipsum' text into the template
\usepackage{graphicx}
\usepackage{caption}
\usepackage{sectsty} % Allows customizing section commands
\allsectionsfont{\centering \normalfont\scshape} % Make all sections centered, the default font and small caps
\usepackage{listings}
\usepackage{color}
\usepackage{fancyhdr} % Custom headers and footers
\usepackage{multicol}
\usepackage{subcaption} %Insert multicolumns plots
\pagestyle{fancyplain} % Makes all pages in the document conform to the custom headers and footers
\usepackage{tikz} %the most complex and powerful tool to create graphic elements in LATEX
\usepackage{url}
\usetikzlibrary{matrix,arrows,fit}
\tikzset{circarrow/.style={
        *->,
        shorten <=-2pt
    }
}
\fancyhead{} % No page header - if you want one, create it in the same way as the footers below
\fancyfoot[L]{} % Empty left footer
\fancyfoot[C]{} % Empty center footer
\fancyfoot[R]{\thepage} % Page numbering for right footer
\renewcommand{\headrulewidth}{0pt} % Remove header underlines
\renewcommand{\footrulewidth}{0pt} % Remove footer underlines
\setlength{\headheight}{13.6pt} % Customize the height of the header

\definecolor{mygreen}{rgb}{0,0.6,0}
\definecolor{mygray}{rgb}{0.5,0.5,0.5}
\definecolor{mymauve}{rgb}{0.58,0,0.82}

\lstset{ %
  backgroundcolor=\color{white},   % choose the background color; you must add \usepackage{color} or \usepackage{xcolor}
  basicstyle=\footnotesize,        % the size of the fonts that are used for the code
  breakatwhitespace=false,         % sets if automatic breaks should only happen at whitespace
  breaklines=true,                 % sets automatic line breaking
  captionpos=b,                    % sets the caption-position to bottom
  commentstyle=\color{mygreen},    % comment style
  frame=single,                    % adds a frame around the code
  keepspaces=true,                 % keeps spaces in text, useful for keeping indentation of code (possibly needs columns=flexible)
  keywordstyle=\color{blue},       % keyword style
  language=Matlab,                 % the language of the code
  numbers=left,                    % where to put the line-numbers; possible values are (none, left, right)
  numbersep=5pt,                   % how far the line-numbers are from the code
  numberstyle=\tiny\color{mygray}, % the style that is used for the line-numbers
  rulecolor=\color{black},         % if not set, the frame-color may be changed on line-breaks within not-black text (e.g. comments (green here))
 tabsize=2,                       % sets default tabsize to 2 spaces
  title=\lstname                   % show the filename of files included with \lstinputlisting; also try caption instead of title
}



\numberwithin{equation}{section} % Number equations within sections (i.e. 1.1, 1.2, 2.1, 2.2 instead of 1, 2, 3, 4)
\numberwithin{figure}{section} % Number figures within sections (i.e. 1.1, 1.2, 2.1, 2.2 instead of 1, 2, 3, 4)
\numberwithin{table}{section} % Number tables within sections (i.e. 1.1, 1.2, 2.1, 2.2 instead of 1, 2, 3, 4)

\setlength\parindent{0pt} % Removes all indentation from paragraphs - comment this line for an assignment with lots of text

%----------------------------------------------------------------------------------------
%    TITLE SECTION
%----------------------------------------------------------------------------------------



\newcommand{\horrule}[1]{\rule{\linewidth}{#1}} % Create horizontal rule command with 1 argument of height
\newcommand{\PP}{\mathcal{P}}

%\title{    
%\normalfont \normalsize 
%\textsc{UT Austin, Mccombs Business School} \\ [25pt] % Your university, school and/or department name(s)
%\horrule{0.5pt} \\[0.4cm] % Thin top horizontal rule
%\huge Homework 4 for Network Optimization\\ % The assignment title
%\horrule{2pt} \\[0.5cm] % Thick bottom horizontal rule
%}

%\author{Long Zhao lz4786} % Your name

%\date{\normalsize\today} % Today's date or a custom date

%\graphicspath{{/Users/zhaolong/Google Drive/Network Optimization/Homework1}}

\begin{document}

%\maketitle % Print the title

\section{About $\eta$}
The equations we have for the function $g$ and $h$ ($V(x,s,q) = g(x,q)\times h(s)$) are
\begin{equation*}
\begin{split}
  &\frac{1}{2} \sigma^2 g_{xx}(x,q) + \kappa (\alpha - x) g_x(x,q) - (\beta + \eta) g(x,q) = 0\\
  &f(s) h'(s) + \eta h(s) = 0 \quad h(0) = 1.
\end{split}
\end{equation*}
The solution to the second ODE is 
\begin{equation}\label{Equation_h}
  h(s) = \exp(-\eta \int_{0}^{s} \frac{1}{f(y)}dy)
\end{equation}

Last time, we mentioned that 
\begin{equation*}
  0 = \frac{\partial V(x,s,q)}{\partial \eta} = \frac{\partial g(x,q;\eta)}{\partial \eta} h(s;\eta) + g(x,q;\eta) \frac{\partial h(s;\eta)}{\partial \eta}
\end{equation*}
By equation (\ref{Equation_h}), we have
\begin{equation*}
  \frac{\partial g(x,q;\eta)}{\partial \eta} h(s;\eta) - g(x,q;\eta) h(s;\eta) \int_{0}^s\frac{1}{f(y)}dy = 0
\end{equation*}
Equation (\ref{Equation_h}) tells us $h>0$, therefore
\begin{equation*}
\begin{split}
&  \frac{\partial g(x,q;\eta)}{\partial \eta} - g(x,q;\eta)\int_{0}^s\frac{1}{f(y)}dy = 0 \\
\Rightarrow \quad & \frac{\frac{\partial g(x,q;\eta)}{\partial \eta}}{g(x,q;\eta)} = \int_{0}^s\frac{1}{f(y)}dy.
\end{split}
\end{equation*}
Right hand side is a function of $s$ while the left hand is not. Contradiction!

\begin{itemize}
  \item $\frac{\partial V(x,s,q)}{\partial \eta} = 0$ doesn't mean that the function V is not a function of $\eta$. Instead, we choose $\eta$ to maximize V.
  \item It is also impossible to have two different $\eta$s to have the same $V$. If not, assume 
  \begin{equation*}
  	g(x,q;\eta_1)h(s;\eta_1) = g(x,q;\eta_2)h(s;\eta_2).
  \end{equation*}
  In this situation, 
  \begin{equation*}
  	\frac{g(x,q;\eta_1)}{g(x,q;\eta_2)} = \frac{h(s;\eta_1)}{h(s;\eta_2)}.
  \end{equation*}
  The left hand side is a function of $x$ and $q$ while the right hand side is a function of $s$ (because $\eta_1$ and $\eta_2$ are given constants). It is impossible to happen. As a result, it is not that
  \begin{equation*}
  	V(x,s,q) = g(x,q;\eta)\times h(s;\eta)
  \end{equation*}
	holds for all $\eta$.
\end{itemize}

\section{Boundary Condition}
The boundary conditions are
\begin{equation*}
  V_q(x,s,q) = \exp(x +s) - \mu(q)
\end{equation*}
and 
\begin{equation*}
  V_q(x,s,q) = \exp(x +s) + \lambda(q).
\end{equation*}
The first one is at the buying boundary while the second one is at the selling boundary.
Plug $g$ and $h$ into previous equations to have
\begin{equation*}
  g_q(x,q) = \frac{\exp(x +s) - \mu(q)}{h(s)}
\end{equation*}
on the selling boundary and
\begin{equation*}
  g_q(x,q)= \frac{\exp(x +s) + \lambda(q)}{h(s)}
\end{equation*}
on the buying boundary. For a fixed $s$ and any $\eta$, the problem becomes the same type as the non-seasonality problem which we can solve effectively. However, two different $s_1$ and $s_2$ are super likely to give different $g(x,q)$ which contradicts the separation of variables at the first place.


It is unlikely to satisfy the boundary condition with all different $s$ with only one $\eta$. As a result, I propose
\begin{equation*}
  V(x,s,q) = \int_{-\infty}^{+\infty}g(x,q;\eta)h(s;\eta)t(\eta)d\eta.
\end{equation*}
I want to use the boundary condition to find an approximation of the function $t$, but I don't know how.



\section{Separation Variable Method in Heat Equation}
\url{http://en.wikipedia.org/wiki/Separation_of_variables}




\end{document}

